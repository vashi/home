\documentclass[paper=a4, fontsize=12pt, titlepage=false, twoside=false]{scrartcl}

%\usepackage{fullpage}
\usepackage{xltxtra}
\usepackage{fontspec}
\usepackage{libertine}
%\defaultfontfeatures{Scale=MatchLowercase, Mapping=tex-text}  %% устанавливает поведение шрифтов по умолчанию
%\setmainfont{Minion}  %% задаёт основной шрифт документа
%\setsansfont{Linux Biolinum}  %% задаёт шрифт без засечек
%\setmonofont{Liberation Mono}  %% задаёт моноширинный шрифт
\usepackage{polyglossia}  %% подключает пакет многоязыкой вёрстки
%\newfontfamily\russianfont{Linux Libertine}

\setdefaultlanguage[spelling=modern]{russian}  %% устанавливает язык по умолчанию
\setotherlanguage{english}

%\usepackage[backend=biber,language=auto]{biblatex}
%\usepackage{csquotes}

% Доп. размеры
\usepackage[12pt]{moresize}

% Настройка PDF
\usepackage[xetex]{hyperref}
\usepackage{pdflscape}
% Цвета для ссылок
\usepackage{xecolor}
%\definecolor{rltblue}{rgb}{0,0,0.75}
\hypersetup{
      colorlinks=true,
%      linkcolor=rltblue,
%      pagebackref,
%      bookmarks=true,
      pdftitle={Технологии перспективной АСУ надводного корабля},
      pdfsubject={Доклад «Технологии перспективной АСУ НК»},
      pdfauthor={Шлыков Василий Александрович},
      pdfkeywords={АСУ} {перспективный технологии} {флот},
}

%\usepackage[a4paper,pass]{geometry}
%\usepackage{dpfloat}

% Нумерация рисунков.
%\usepackage{chngcntr}
%\counterwithout{figure}{chapter}

\usepackage{enumitem} % Модификация списков
% Маркер ненумерованных списков
\renewcommand{\labelitemi}{–}
\setlist{nosep}

% Отступ первой строки абзаца
\setlength{\parindent}{1cm}
\usepackage{indentfirst}
%\usepackage{parskip}
%\setlength{\parskip}{2mm}

\usepackage{titlesec}
%\titleformat{\chapter}{\HUGE\sffamily}{}{0pt}{}{}
\titleformat{\section}{\Large\sffamily}{\thesection}{4mm}{}{}
\titlespacing{\paragraph}{1cm}{0mm}{0mm}

% Нормальные пробелы
\frenchspacing

% Колонтитулы
\usepackage{fancyhdr}
\pagestyle{fancy}
\lhead{\textit{\xecolor{gray} Защищенная вычислительная платформа «Глобула»}}
\chead{}
%\renewcommand{\chaptermark}[1]{\markboth{#1}{}}
\rhead{\textit{\xecolor{gray} \leftmark}}
%\rhead{\textit{\xecolor{gray} ОАО «Северное ПКБ»}}
\renewcommand{\headrulewidth}{0mm}

% Переносы
\pretolerance=1000
\hyphenpenalty=1000
\tolerance=2000

\title{Отчет}
\author{Шлыков Василий Александрович}
\date{}

\newcommand{\Cloud}{«Глобула» } 
\newcommand{\Cloudi}{«Глобулы» } 
\newcommand{\Zop}{Защищенная облачная платформа }
\newcommand{\zop}{защищенная облачная платформа }
\newcommand{\astrase}{«Astra Linux Special Edition» }
\newcommand{\astrace}{«Astra Linux CE» }
\newcommand{\underlin}{\underline{\phantom{i}}}

\begin{document}

\begin{abstract}
Рассматриваются требования к перспективным АСУ надводных кораблей,
существующие проблемы создания АСУ. На основе анализа существующих
отечественных подходов и анализа современных и перспективных
зарубежных разработок даются предложения по технологиям проектирования,
разработки, а также базовым технологиям перспективной АСУ.
\end{abstract}

\pagebreak

\section{Введение}

Требования предъявляемые к перспективной АСУ, ставят планку в вопросах
функциональной насыщенности, обеспечения нефункциональных характеристик.

Перспективная АСУ, помимо традиционных задач ведения всех видов обстановки,
управления боевым использованием вооружения, выработка рекомендаций и т.\,п.,
должна интегрировать следующие задачи:
\begin{itemize}
  \item управление техническими средствами корабля;
  \item мониторинг (распознавание нештатных ситуаций, дистанционное наблюдение,
        видеорегистрация);
  \item охрана помещений, контроль доступа и определение местоположения экипажа;
  \item управление борьбой за живучесть;
  \item интегрированная логистическая поддержка;
  \item обеспечение группового взаимодействия;
  \item учебно-тренировочный режим охватывающий все системы корабля;
  \item аппаратная унификация и единый интерфейс оператора;
  \item и д.\,р.
\end{itemize}

Реализовать поставленные задачи невозможно без внедрения современных технологий
на всех уровнях и этапах жизненного цикла АСУ.

В данной работе предлагаются технологии организационного и технического плана,
необходимые для решения поставленных задач. Некоторые из предлагаемым технологий
были успешно апробированы в ряде проектов с участием
Северного проектно-конструкторского бюро, а часть предлагается на основе зарубежного
опыта построения систем аналогичного класса.

\section{Существующие проблемы}

Необходимо обозначить следующие организационно-технические проблемы:
\begin{itemize}
  \item отсутствие централизованного проектирования и выработки согласованных
        требований;
  \item отсутствие опыта и технологий организации разработки ПО в широкой кооперации;
  \item отсутствие современных утвержденных стандартов проектирования, разработки
        и тестирования ПО;
  \item повсеместное использование в разработке технологий, не отвечающих современным
        требованиям и подходам;
  \item отсутствие квалифицированного тестирования и испытаний.
\end{itemize}

Технические проблемы:
\begin{itemize}
 \item отсутствие аппаратной и программной унификации по разным проектам;
 \item долгая и тяжелая интеграция на корабле;
 \item дублирование ПО: основной, резервный, автономный режимы;
 \item непрозрачность для заказчика.
\end{itemize}

Отсутствие аппаратной и программной унификации пагубно сказывается на
возможности модернизации, ухудшает переносимость, затрудняют эксплуатацию,
увеличивают ЗиП.

Проблемы системной интеграции и испытаний происходят в следствие отсутствия
сколь-либо системного подхода к обеспечению и контролю качества ПО,
слабости испытательной базы. Трехкратное дублирование ПО
в условиях острой нехватки профильных специалистов усугубляет
данную проблему, а кроме того, увеличивает стоимость и сроки сдачи проекта.

На основе сопоставления сроков и на качества образцов программного обеспечения
разработанных для существующих НК, следует говорить о том, что существующие подходы
не обеспечивают необходимых требований даже на существующих системах. В силу
широты охвата функциональных задач перспективной АСУ, для успешной реализации
проекта, необходимо совершить качественный скачок в технологиях на всех стадиях
жизненного цикла ПО.

\section{Зарубежный опыт}

В середине 90-ых годов на западе столкнулись с аналогичными проблемами в области
разработки программного обеспечения военного назначения.
В результате в течение десятилетия были пересмотрены все процессы и технологии,
связанные с разработкой ПО \cite{report94,report98,report00}.

Одними из технических решений были переход на открытую архитектуру
\cite{report98,guidesis,oanaval} и максимальное
привлечение промышленных технологий (COTS) \cite{cots}.

Главный принцип открытой архитектуры -- в утвердительном ответе на
вопрос: «Может ли сторонний квалифицированный участник заменить
компонент системы, используя только доступные технические и
функциональные спецификации компонента?» \cite{ibmoa}.

ОА декларирует следующие принципы и требования к программному обеспечению
\cite{oanaval,ibmoa}:
\begin{itemize}
  \item модульная архитектура;
  \item широко распространенные промышленные стандарты;
  \item стандартное промышленное оборудование;
  \item серийное программное обеспечение.
\end{itemize}

75 Вирджинии.

\section{IEEE 12207}

Одним из примеров реализации современных технологий разработки
ПО служит совместный российско-индийская
боевая информационно-управляющая система (БИУС) фрегата проекта 17 (БИУС-17).
Важнейшей причиной успешности данного проекта следует отметить
использования стандарта разработки IEEE 12207 \cite{ieee12207}.

Первая версия стандарта разработана в 1998 году, согласно стратегии
замещения военных стандартов промышленными \cite{perry,report94} был
утвержден минобороны США в 1998 году на замену MIL-STD-498. В России
часть стандарта была утверждена как ГОСТ Р ИСО/МЭК 12207-2010.

На примере БИУС-17 показано, что данный стандарт позволяет:
\begin{itemize}
  \item эффективно организовать широкую кооперацию;
  \item снизить риски интеграции;
  \item повысить качество программного обеспечения;
  \item перейти к производству линейки ПО (серийной производство).
\end{itemize}

Стандарт IEEE 12207 состоит из трех частей: 12207.0, 12207.1, 12207.2
и множества сопутствующих стандартов \cite{impl12207} образующих
семейство самосогласованных стандартов жизненного цикла ПО. 
Разрабатывая ПО в соответствии с данным набором стандартов,
возможно построение систем соответствующих принципу открытой
архитектуры за счет формализованного и детализированного описания
процессов и спецификаций каждого этапа.

Полученная после проектирования и разработки
документация на систему позволяет
применять современные методики оценки трудоемкости, такие как
модель издержек разработки (COCOMO) \cite{cocomo2}.
Подобные методики позволяют рассчитывать оценку
трудоемкости разработки модернизированной системы.

Следует считать использование семейства стандартов
IEEE 12207 как необходимое условие разработки перспективной АСУ.

%\subsection{Информационное пространство проектирования}

%\subsection{Управление качеством}

\section{Аппаратная архитектура}

В силу тесной интеграции функциональных
задач по контролю и использованию всех технических
систем корабля в единой АСУ, необходимо тщательно продумать
сетевую топологию и аппаратную архитектуру АСУ.

В первую очередь, внушительно
расширяется номенклатура циркулирующей информации, количество и тип
абонентов. Решение проблемы традиционным способом — одноранговой
сетью, представляется трудновыполнимым в силу следующих причин:
\begin{enumerate}
  \item высокая протяженность и большое количество аппаратуры сопряжения
        неизбежно приведет к резкому возрастанию сложности и уменьшению
        живучести;
  \item сложность обеспечения приоритетов в обслуживании различным
        классам информации;
  \item существующие промышленные технологии решения задач,
        обозначенных в ТЗ АСУ: мониторинг, противопожарная защита,
        охранная сигнализация, живучесть, управление техническими
        средствами не используют Ethernet в качестве технологии передачи
        данных.
\end{enumerate}

В связи с обозначенными проблемами выглядит перспективным использование
сетевой структуры общекорабельной !!! (TSCE) \cite{tsce}. Суть рассматриваемых
решений состоит в разделении вычислительной сети на два уровня: магистральная
сеть Ethernet 10/40 Гбит/с и «инженерный» уровень на базе технологий
промышленной сети.

Топология магистрального уровня представляет собой сдвоенное кольце на
оптоволокне с резервированием на базе технологии Ethernet. Структура магистрального
хорошо известна и используется в России и за рубежом \cite{gedms,gedmsv2}.

Инженерный уровень предназначен для объединения датчиков, контроллеров и
исполнительных устройств между собой, а также с высокоуровневой АСУ.
Оборудование предоставляет различные интерфейсы промышленной сети: CAN,
Profibus, MODBUS и др. Инженерный уровень также резервирован и локализован
в пределах одной пожарной зоны. Образуемые сегменты связаны через
маршрутизаторы/шлюзы CAT/Ethernet, Profinet между собой по
волоконно-оптическим линиям.

Уровни связаны между собой через аппаратно-программные шлюзы с поддержкой
протоколов MODBUS/TCP, Profinet и др. Общая схема сети показана на \ris{network}.

Большая часть необходимого оборудования выпускается серийно отечественной
промышленностью.

\subsection{Аппаратно-программная архитектура}

В силу высокой интеграции всех функциональных подсистем в рамках единой
АСУ первоочередной задачей является обеспечение надежности, готовности и
безопасности аппаратно-программного комплекса.

В существующих системах автоматизации («Линкор-22350», «Сигма» и др.)
обеспечение надежности осуществляется за счет дублирования систем 
управления (СУ). Каждая система может функционировать в одном из трех
режимов: основной, резервный, автономный. В основном режиме СУ
является БИУС, в резервном -- одна из систем вооружения, в автономном
режиме используется встроенная СУ.

Для каждого из режимов ПО СУ разрабатывается независимо и обеспечивает
интероперабельность в рамках только одного проекта, что
приводит к множественным проблемам проектирования, разработки и
эксплуатации.

В отличии от вышеперечисленных систем, в БИУС-17 надежность и
готовность обеспечивается: на аппаратном уровне -- двукратное
резервирование (горячее и холодное), на программном уровне --
применением точек синхронизации на уровне прикладного ПО.
Такая схема позволила отказаться от резервного режима и сосредоточиться
на основном (централизованном) и автономном режимах.

Несмотря на достоинства такой архитектуры, она обладает рядом недостатков.
Реализация точек синхронизации лежит на прикладном уровне, т\.е.
требует вмешательства в каждую функциональную задачу.

В силу ряда факторов (номенклатура задач, широта необходимой кооперации)
подобная схема труднореализуема в перспективной АСУ, поэтому
в качестве перспективной рассматривается архитектура на основе
кластера высокой готовности (КВГ).

Все вычислительные средства корабля необходимо проранжировать по
критерию обеспечения живучести: критические, важные, обеспечивающие
(второстепенные).

Для критических систем необходимо предоставить двукратное резервирование.
Горячее резервирование обеспечивается КВГ, холодное -- дублированием
КВГ с пространственным разнесением по кораблю. Синхронизация осуществляется
на системном уровне, например с помощью функции гео-репликации
распределенных файловых систем.

Для важных и второстепенных систем надежность обеспечивается однократным
резервированием в КВГ, в зависимости от заданных требований по надежности
и готовности.

Аппаратный комплекс должен строиться на базе серийного
промышленного оборудования (COTS) и программного обеспечения.

\section{Единое информационное пространство}

В настоящее время, корабль не проектируется в разрезе
информационной системы. Вместо этого задача ставится как
«сопряжение систем вооружения корабля». Такая
постановка ведет к тому, что важнейшим задачам по созданию системных
требований, анализу алгоритмов взаимодействия уделяется внимание по
остаточному принципу.

Одной из новаций, необходимой для воплощения перспективной АСУ НК,
является переход к проектированию корабля как информационной
системы. Такая система изначально должна создаваться
на принципах открытой архитектуры. 

В практической плоскости это означает отказ
от тесно связанных систем на основе парных протоколов сопряжения
в силу недостатков такой архитектуры \cite{p2p}.

в пользу компонентной архитектуры
на принципах единого информационного пространства \cite{distrib,dds} (\ris{event}).

\subsection{Многоуровневая архитектура}

\section{Единая имитационная среда}

\section{Атрибуты качества архитектуры}

\section{Заключение}

\section{Литература}

\end{document}